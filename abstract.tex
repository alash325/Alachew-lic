%%% Abstract.tex --- 
%% 
%% Filename: Abstract.tex
%% Description: 
%% Author: Ola Leifler
%% Maintainer: 
%% Created: Thu Oct 14 13:34:11 2010 (CEST)
%% Version: $Id$
%% Version: 
%% Last-Updated: Tue Dec  1 15:19:52 2015 (+0100)
%%           By: Ola Leifler
%%     Update #: 4
%% URL: 
%% Keywords: 
%% Compatibility: 
%% 
%%%%%%%%%%%%%%%%%%%%%%%%%%%%%%%%%%%%%%%%%%%%%%%%%%%%%%%%%%%%%%%%%%%%%%
%% 
%%% Commentary: 
%% 
%% 
%% 
%%%%%%%%%%%%%%%%%%%%%%%%%%%%%%%%%%%%%%%%%%%%%%%%%%%%%%%%%%%%%%%%%%%%%%
%% 
%%% Change log:
%% Completed Language Reading MBS
%% 
%% RCS $Log$
%%%%%%%%%%%%%%%%%%%%%%%%%%%%%%%%%%%%%%%%%%%%%%%%%%%%%%%%%%%%%%%%%%%%%%
%% 
%%% Code:

Model-based tools and methods are playing important roles in the design and analysis of cyber-physical systems before building and testing physical prototypes. The development of increasingly complex CPSs requires the use of multiple tools for different phases of the development lifecycle, which in turn depends on the ability of the supporting tools to interoperate. However, currently no vendor provides comprehensive end-to-end systems engineering tool support across the entire product lifecycle, and no mature solution currently exists for integrating different system modeling and simulation languages, tools and algorithms in the CPSs design process. Thus, modeling and simulation tools are still used separately in industry. 

The unique challenges in integration of CPSs are a result of the increasing heterogeneity of components and their interactions, increasing size of systems, and essential design requirements from various stakeholders. The corresponding system development involves several specialists in different domains, often using different modeling languages and tools. In order to address the challenges of CPSs and facilitate design of system architecture and design integration of different models, significant progress needs to be made towards model-based integration of multiple design tools, languages, and algorithms into a single integrated modeling and simulation environment. 

In this thesis we present the need for methods and tools with the aim of developing techniques for numerically stable co-simulation, advanced simulation model analysis, simulation-based optimization, and traceability capability, and making them more accessible to the model-based cyber physical product development process, leading to more efficient simulation. In particular, the contributions of this thesis are as follows: 1) development of a model-based dynamic optimization approach by integrating optimization into the model development process; 2) development of a graphical co-modeling editor and co-simulation framework for modeling, connecting, and unified system simulation of several different modeling tools using the TLM technique; 3) development of a tool-supported method for multidisciplinary collaborative modeling and traceability support throughout the development process for CPSs; 4) development of an advanced simulation modeling analysis tool for more efficient simulation.


%%%%%%%%%%%%%%%%%%%%%%%%%%%%%%%%%%%%%%%%%%%%%%%%%%%%%%%%%%%%%%%%%%%%%%
%%% Abstract.tex ends here


%%% Local Variables: 
%%% mode: latex
%%% TeX-master: "demothesis"
%%% End: 

